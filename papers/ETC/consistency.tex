\begin{coqdoccode}
\coqdocemptyline
\coqdocemptyline
\end{coqdoccode}
\section{Introduction}


A crucial property of a logical system is consistency, which states that it does not
entails a contradiction. Basically, consistency implies that not all formulas
are provable.  While having a simple motivation, consistency proofs rely on
the well-known admissibility of cut property, which has a complex inductive proof.
Gentzen, in his seminal work, gives the first consistency proof of logic by introducing an
auxiliar formalism, the sequent calculus, in which consistency is trivial. Next, Gentzen showed
that natural deduction system is equivalent to his sequent calculus extended with an
additional rule: the cut rule. The final (and hardest) piece of Gentzen's proof is to 
show that the cut rule is redundant, i.e., it is admissible. As a consequence, we know
something stronger: all propositions provable in the natural deduction system are also provable
in the sequent calculus without cut. Since we know that the sequent calculus is consistent,
we hence also know that the natural deduction calculus is~\cite{Negri2001}.


However, proving the admissibility of cut is not easy, even for simple logics.
Proofs of admissibility need nested inductions and we need to be really careful to
ensure a decreasing measure on each use of the inductive hypothesis. Such proofs have
a heavy syntactic flavor since they recursively manipulate proof tree structures to
eliminate cuts. A more semantic based approach relies on interpreting logics as its
underlying $\lambda$-calculus and prove consistency by using its computation machinery.
In this work, we report the Coq formalization of these two approaches and advocate the use
of the latter since it result on easy to follow proofs. We organize this work
as follows: Section \ref{sec:definitions} present basic definitions about the logic considered
and Section \ref{sec:semantics} describe the semantics of our logic objects and its
consistency proof. Section \ref{sec:conclusion} presents a brief comparision between
two consistency proofs and concludes.


The complete formalization was verified using Coq version 8.10.2 and its available
on-line~\cite{Sasdelli20}. For space reasons, we rely on reader's intuition to
explain Coq code fragments. Good introductions to Coq are available elsewhere~\cite{Chlipala13}.


\section{Basic Definitions}\label{sec:definitions}


First, we consider formulas of a minimal fragment of propositional logics which is formed only by the constant
\emph{falsum} ($\bot$) and logic implication ($\supset$). Following common
practice, we denote contexts by a list of formulas. The following Coq snipetts
declare these concepts.


\begin{minipage}[c]{0.3\textwidth}
\begin{coqdoccode}
\coqdocemptyline
\coqdocnoindent
\coqdockw{Inductive} \coqdocvar{$\alpha$} : \coqdockw{Set} :=\coqdoceol
\coqdocnoindent
\ensuremath{|} \coqdocvar{Falsum} : \coqdocvar{$\alpha$}\coqdoceol
\coqdocnoindent
\ensuremath{|} \coqdocvar{Implies} : \coqdocvar{$\alpha$} \ensuremath{\rightarrow} \coqdocvar{$\alpha$} \ensuremath{\rightarrow} \coqdocvar{$\alpha$}.\coqdoceol
\coqdocemptyline
\coqdocnoindent
\coqdockw{Definition} \coqdocvar{$\Gamma$} := \coqdocvar{list} \coqdocvar{$\alpha$}.\coqdoceol
\coqdocemptyline
\end{coqdoccode}
\end{minipage}
\begin{minipage}[c]{0.6\textwidth}
While types for formulas (\coqdocvar{$\alpha$}) and contexs (\coqdocvar{$\Gamma$}) have an immediate interpretation, the previous types
miss an important part of propositional logic: variables. We represent variables by an inductive judgement
which states the membership of a formula in a context.
\end{minipage}


We let notation $\alpha \in \Gamma$ denote an inductive predicate that states that a formula $\alpha$
is an element of context $\Gamma$. The rules for variable judgement and its Coq implementation are presented below.


\begin{minipage}[c]{0.6\textwidth}
 \begin{coqdoccode}
\coqdocnoindent
\coqdockw{Inductive} \coqdocvar{var} : \coqdocvar{$\Gamma$} \ensuremath{\rightarrow} \coqdocvar{$\alpha$} \ensuremath{\rightarrow} \coqdockw{Type} :=\coqdoceol
\coqdocnoindent
\ensuremath{|} \coqdocvar{Here} : \coqdockw{\ensuremath{\forall}} \coqdocvar{G} \coqdocvar{p}, \coqdocvar{var} (\coqdocvar{p} :: \coqdocvar{G}) \coqdocvar{p}\coqdoceol
\coqdocnoindent
\ensuremath{|} \coqdocvar{There} : \coqdockw{\ensuremath{\forall}} \coqdocvar{G} \coqdocvar{p} \coqdocvar{p'}, \coqdocvar{var} \coqdocvar{G} \coqdocvar{p} \ensuremath{\rightarrow} \coqdocvar{var} (\coqdocvar{p'} :: \coqdocvar{G}) \coqdocvar{p}.\coqdoceol
\end{coqdoccode}
\end{minipage}
\begin{minipage}[c]{0.3\textwidth}
\[
\begin{array}{c}
\infer[_{\{Here\}}]{\alpha \in (\alpha :: \Gamma)}{}\\ \\
\infer[_{\{There\}}]{\alpha \in (\beta :: \Gamma)}{\alpha \in \Gamma} \\ \\
\end{array}
\]
\end{minipage}
The first constructor of type \coqdocvar{var} specifies that a formula $\alpha$ is in the context $\alpha :: \Gamma$ and
the constructor \coqdocvar{There} specifies that if a formula $\alpha$ is in $\Gamma$ then we have
 $\alpha \in (\beta :: \Gamma)$, for any formula $\beta$.


Using the previous definitions, we can implement natural deduction rules for our minimal logic, as presented below.
\begin{coqdoccode}
\coqdocemptyline
\coqdocemptyline
\end{coqdoccode}
\begin{minipage}[c]{0.6\textwidth}
\begin{coqdoccode}
\coqdocnoindent
\coqdockw{Inductive} \coqdocvar{nd} : \coqdocvar{$\Gamma$} \ensuremath{\rightarrow} \coqdocvar{$\alpha$} \ensuremath{\rightarrow} \coqdockw{Type} :=\coqdoceol
\coqdocnoindent
\ensuremath{|} \coqdocvar{Id} : \coqdockw{\ensuremath{\forall}} \coqdocvar{G} \coqdocvar{p},\coqdoceol
\coqdocindent{2.00em}
\coqdocvar{var} \coqdocvar{G} \coqdocvar{p} \ensuremath{\rightarrow}\coqdoceol
\coqdocindent{2.00em}
\coqdocvar{nd} \coqdocvar{G} \coqdocvar{p}\coqdoceol
\coqdocnoindent
\ensuremath{|} \coqdocvar{ExFalsum} : \coqdockw{\ensuremath{\forall}} \coqdocvar{G} \coqdocvar{p},\coqdoceol
\coqdocindent{2.00em}
\coqdocvar{nd} \coqdocvar{G} \coqdocvar{Falsum} \ensuremath{\rightarrow}\coqdoceol
\coqdocindent{2.00em}
\coqdocvar{nd} \coqdocvar{G} \coqdocvar{p}\coqdoceol
\coqdocnoindent
\ensuremath{|} \coqdocvar{Implies\_I} : \coqdockw{\ensuremath{\forall}} \coqdocvar{G} \coqdocvar{p} \coqdocvar{p'},\coqdoceol
\coqdocindent{2.00em}
\coqdocvar{nd} (\coqdocvar{p'} :: \coqdocvar{G}) \coqdocvar{p} \ensuremath{\rightarrow}\coqdoceol
\coqdocindent{2.00em}
\coqdocvar{nd} \coqdocvar{G} (\coqdocvar{Implies} \coqdocvar{p'} \coqdocvar{p})\coqdoceol
\coqdocnoindent
\ensuremath{|} \coqdocvar{Implies\_E} : \coqdockw{\ensuremath{\forall}} \coqdocvar{G} \coqdocvar{p} \coqdocvar{p'},\coqdoceol
\coqdocindent{2.00em}
\coqdocvar{nd} \coqdocvar{G} (\coqdocvar{Implies} \coqdocvar{p'} \coqdocvar{p}) \ensuremath{\rightarrow}\coqdoceol
\coqdocindent{2.00em}
\coqdocvar{nd} \coqdocvar{G} \coqdocvar{p'} \ensuremath{\rightarrow}\coqdoceol
\coqdocindent{2.00em}
\coqdocvar{nd} \coqdocvar{G} \coqdocvar{p}.\coqdoceol
\end{coqdoccode}
\end{minipage}
\begin{minipage}[c]{0.3\textwidth}
\[
\begin{array}{c}
\infer[_{\{Id\}}]{\Gamma \vdash x}{x \in \Gamma} \\ \\
\infer[_{\{Ex\}}]{\Gamma \vdash \alpha}{\Gamma \vdash \bot}\\ \\
\infer[_{\{\supset-I\}}]{\Gamma \vdash \alpha \supset \beta}{\Gamma \cup \{\alpha\} \vdash \beta}\\ \\
\infer[_{\{\supset-E\}}]{\Gamma \vdash \beta}{\Gamma \vdash \alpha \supset \beta & \Gamma \vdash \alpha}
\end{array}
\]
\end{minipage}


The first rule (\coqdocvar{Id}) stabilishes that any formula in the context is provable and rule \coqdocvar{ExFalsum} defines
the principle \emph{ex-falsum quod libet} which allow us to prove any formula if we have a deduction of \coqdocvar{Falsum}.
Rule \coqdocvar{Implies\_I} specifies that from a deduction of a formula \coqdocvar{p} from a context \coqdocvar{p'} :: \coqdocvar{G}, \coqdocvar{nd} (\coqdocvar{p'} :: \coqdocvar{G}) \coqdocvar{p},
we can prove the implication \coqdocvar{Implies} \coqdocvar{p'} \coqdocvar{p}. The last rule, \coqdocvar{Implies\_E}, represents the well-known \emph{modus-ponens},
which allows us to deduce a formula \coqdocvar{p} from deductions of \coqdocvar{Implies} \coqdocvar{p'} \coqdocvar{p} and \coqdocvar{p'}.


Next section uses the relation between logics and $\lambda$-calculus and its evaluation to prove the consistency of
minimal logic.
\begin{coqdoccode}
\coqdocemptyline
\coqdocemptyline
\end{coqdoccode}
 \section{Semantics and Consistency}\label{sec:semantics}


We prove the consistency of logics by exploring its correspondence with the simply typed
$\lambda$-calculus. We do this by implementing in Coq a well-known idea ~\cite{Augustsson99anexercise}
for implementating denotational semantics for $\lambda$-term in type theory based proof assistants.


We define the denotation of a formula by recursion on its structure. The idea is to associate the
empty type (\coqdocvar{False}) with the formula \coqdocvar{Falsum} and a function type with formula \coqdocvar{Implies} \coqdocvar{p1} \coqdocvar{p2},
as presented next.
 \begin{coqdoccode}
\coqdocemptyline
\coqdocnoindent
\coqdockw{Program Fixpoint} \coqdocvar{sem\_form} (\coqdocvar{p} : \coqdocvar{$\alpha$}) : \coqdockw{Type} :=\coqdoceol
\coqdocindent{1.00em}
\coqdockw{match} \coqdocvar{p} \coqdockw{with}\coqdoceol
\coqdocindent{1.00em}
\ensuremath{|} \coqdocvar{Falsum} \ensuremath{\Rightarrow} \coqdocvar{False} \ensuremath{|} \coqdocvar{Implies} \coqdocvar{p1} \coqdocvar{p2} \ensuremath{\Rightarrow} \coqdocvar{sem\_form} \coqdocvar{p1} \ensuremath{\rightarrow} \coqdocvar{sem\_form} \coqdocvar{p2}\coqdoceol
\coqdocindent{1.00em}
\coqdockw{end}.\coqdoceol
\coqdocemptyline
\end{coqdoccode}
Using \coqdocvar{sem\_form} function, we can define context semantics as tuples
of formula semantics as follows:
 \begin{coqdoccode}
\coqdocemptyline
\coqdocnoindent
\coqdockw{Program Fixpoint} \coqdocvar{sem\_ctx} (\coqdocvar{G} : \coqdocvar{$\Gamma$}) : \coqdockw{Type} :=\coqdoceol
\coqdocindent{1.00em}
\coqdockw{match} \coqdocvar{G} \coqdockw{with}\coqdoceol
\coqdocindent{1.00em}
\ensuremath{|} $\emptyset$ \ensuremath{\Rightarrow} \coqdocvar{unit} \ensuremath{|} (\coqdocvar{t} :: \coqdocvar{G'}) \ensuremath{\Rightarrow} \coqdocvar{sem\_form} \coqdocvar{t} \ensuremath{\times} \coqdocvar{sem\_ctx} \coqdocvar{G'}\coqdoceol
\coqdocindent{1.00em}
\coqdockw{end}.\coqdoceol
\end{coqdoccode}
Function \coqdocvar{sem\_ctx} recurses over the structure of the input context building
right-nested tuple ending with the Coq \coqdocvar{unit} type, which is a type with a
unique element. Since context are mapped intro tuples, variables must be
mapped into projections on such tuples. This would allow us to retrieve the
value associated with a variable in a context.
\begin{coqdoccode}
\coqdocemptyline
\coqdocnoindent
\coqdockw{Program Fixpoint} \coqdocvar{sem\_var} \{\coqdocvar{G} \coqdocvar{p}\}(\coqdocvar{v} : \coqdocvar{var} \coqdocvar{G} \coqdocvar{p}) : \coqdocvar{sem\_ctx} \coqdocvar{G} \ensuremath{\rightarrow} \coqdocvar{sem\_form} \coqdocvar{p} :=\coqdoceol
\coqdocindent{2.00em}
\coqdockw{match} \coqdocvar{v} \coqdockw{with}\coqdoceol
\coqdocindent{2.00em}
\ensuremath{|} \coqdocvar{Here} \ensuremath{\Rightarrow} \coqdockw{fun} \coqdocvar{env} \ensuremath{\Rightarrow} \coqdocvar{fst} \coqdocvar{env} \ensuremath{|} \coqdocvar{There} \coqdocvar{v'} \ensuremath{\Rightarrow} \coqdockw{fun} \coqdocvar{env} \ensuremath{\Rightarrow} \coqdocvar{sem\_var} \coqdocvar{v'} (\coqdocvar{snd} \coqdocvar{env})\coqdoceol
\coqdocindent{2.00em}
\coqdockw{end}.\coqdoceol
\coqdocemptyline
\end{coqdoccode}
Function \coqdocvar{sem\_var} receives a variable (value of type \coqdocvar{var} \coqdocvar{G} \coqdocvar{p}) and a semantics
of a context (a value of type \coqdocvar{sem\_ctx} \coqdocvar{G}) and returns the value of the formula
represented by such variable. Whenever the variable is built using constructor \coqdocvar{Here},
we just return the first component of the input context semantics and when we have
the constructor \coqdocvar{There} we just call  \coqdocvar{sem\_var} recursively.


Our next step is to define the semantics of natural deduction proofs. The semantics of
proofs is implemented by function \coqdocvar{sem\_nat\_ded} which maps proofs (values of type \coqdocvar{nat\_ded} \coqdocvar{G} \coqdocvar{p})
and context semantics (values of type \coqdocvar{sem\_ctx} \coqdocvar{G}) to the value of input proof conclusion
(type \coqdocvar{sem\_form} \coqdocvar{p}). The first case specifies that the semantics of an identity rule proof
(constructor \coqdocvar{Id}) is just retrieving the value of the underlying variable in the context semantics
by calling function \coqdocvar{sem\_var}. Second case deals with \coqdocvar{ExFalsum} rule: we recurse over the proof
object \coqdocvar{Hf} which will produces a Coq object of type \coqdocvar{False}, which is empty and so we can finish the
definition with an empty pattern match. Semantics of implication introduction (\coqdocvar{Implies\_I}) simply
recurses on the subderivation \coqdocvar{Hp} using an extended context (\coqdocvar{v'} , \coqdocvar{env}). Finally, we define the
semantics of implication elimination as simply function application of the results of the
recursive call on its two subderivations.
\begin{coqdoccode}
\coqdocemptyline
\coqdocnoindent
\coqdockw{Program Fixpoint} \coqdocvar{sem\_nat\_ded} \{\coqdocvar{G} \coqdocvar{p}\}(\coqdocvar{H} : \coqdocvar{nat\_ded} \coqdocvar{G} \coqdocvar{p})\coqdoceol
\coqdocindent{1.00em}
: \coqdocvar{sem\_ctx} \coqdocvar{G} \ensuremath{\rightarrow} \coqdocvar{sem\_form} \coqdocvar{p} :=\coqdoceol
\coqdocindent{1.00em}
\coqdockw{match} \coqdocvar{H} \coqdockw{with}\coqdoceol
\coqdocindent{1.00em}
\ensuremath{|} \coqdocvar{Id} \coqdocvar{v} \ensuremath{\Rightarrow} \coqdockw{fun} \coqdocvar{env} \ensuremath{\Rightarrow} \coqdocvar{sem\_var} \coqdocvar{v} \coqdocvar{env}\coqdoceol
\coqdocindent{1.00em}
\ensuremath{|} \coqdocvar{ExFalsum} \coqdocvar{Hf} \ensuremath{\Rightarrow} \coqdockw{fun} \coqdocvar{env} \ensuremath{\Rightarrow}\coqdoceol
\coqdocindent{3.00em}
\coqdockw{match} \coqdocvar{sem\_nat\_ded} \coqdocvar{Hf} \coqdocvar{env} \coqdockw{with}\coqdoceol
\coqdocindent{3.00em}
\coqdockw{end}\coqdoceol
\coqdocindent{1.00em}
\ensuremath{|} \coqdocvar{Implies\_I} \coqdocvar{Hp} \ensuremath{\Rightarrow} \coqdockw{fun} \coqdocvar{env} \coqdocvar{v'} \ensuremath{\Rightarrow} \coqdocvar{sem\_nat\_ded} \coqdocvar{Hp} (\coqdocvar{v'} , \coqdocvar{env})\coqdoceol
\coqdocindent{1.00em}
\ensuremath{|} \coqdocvar{Implies\_E} \coqdocvar{Hp} \coqdocvar{Ha} \ensuremath{\Rightarrow} \coqdockw{fun} \coqdocvar{env} \ensuremath{\Rightarrow} (\coqdocvar{sem\_nat\_ded} \coqdocvar{Hp} \coqdocvar{env}) (\coqdocvar{sem\_nat\_ded} \coqdocvar{Ha} \coqdocvar{env})\coqdoceol
\coqdocindent{1.00em}
\coqdockw{end}.\coqdoceol
\coqdocemptyline
\end{coqdoccode}
Using all thoses previously defined pieces we can prove the consistency of our little natural
deduction system merely by showing that it should not be the case that we have a proof of \coqdocvar{Falsum}
using the empty set of assumptions. We can proof such fact by exhbiting a term of type
\coqdocvar{nat\_ded} $\emptyset$ \coqdocvar{Falsum} \ensuremath{\rightarrow} \coqdocvar{False}\footnote{Here we use the fact that $\neg \alpha$ is
equivalent to $\alpha \supset \bot$.}, which is trivially done by using function \coqdocvar{sem\_nat\_ded}.
 \begin{coqdoccode}
\coqdocemptyline
\coqdocnoindent
\coqdockw{Theorem} \coqdocvar{consistency} : \coqdocvar{nat\_ded} $\emptyset$ \coqdocvar{Falsum} \ensuremath{\rightarrow} \coqdocvar{False} := \coqdockw{fun} \coqdocvar{p} \ensuremath{\Rightarrow} \coqdocvar{sem\_nat\_ded} \coqdocvar{p} \coqdocvar{tt}.\coqdoceol
\coqdocemptyline
\end{coqdoccode}
\section{Conclusion}\label{sec:conclusion}


In this work we briefly describe a Coq formalization of a semantics based consistency proof for
minimal propositional logic. The complete proof is only 85 lines long and only use of some basic
dependently typed programming features of Coq. We also
formalize the consistency of this simple logic in Coq using Gentzen's admissibility of cut approach
which resulted in around 270 lines of code and uses some extra proof tactics libraries.
As future work, we intend to extend the current formalization to full propositional logic and also
other formalisms like Hilbert systems and analytic tableaux~\cite{smullyan1995first}.


 \begin{coqdoccode}
\end{coqdoccode}
